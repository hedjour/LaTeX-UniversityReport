%Tapé ici votre résumé directement simplement.
\begin{singlespace}
From aparental care to communal breeding, a diversity of partnership can be found amongst animals. Birds are unique among vertebrates in that biparental care is the norm. But biparental care goes along with potential conflicts between patners, where both try to minimize cost of reproduction but must cooperate at the same time. In many bird species, the post-guard phase is an energetically costly period when crucial decisions between costs and benefits have to be made.  
This study on King Penguins carries out the analysis of \textsc{Antavia}'s data, an automatic identification system, during post-guard phases of the reproductive cycle of 40 pairs in 2004-2005. Three different periods were distinguished 1\-- Crèche 1, from thermal emancipation to winter, 2\-- Crèche 2, winter period, 3\-- Crèche 3, refeeding phase until fledging.
Males completed more shifts ashore than females on average during the whole post-guard. This held true in the first two crèches. In the third one, the number of sojourns ashore is not sex-specific anymore but dependent from the cessation in food provisioning. Despite a majority of simultaneous stops, 28\% of chicks received mono-parental care during their last rearing month.  As predicted by the game theory, we observed a partial compensation of the remaining parent. Finally, we suggest that males play a key role in chick provisioning, as only males’ frequency affected the reproductive issue. 

\end{singlespace}