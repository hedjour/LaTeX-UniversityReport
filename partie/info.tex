	%%%%%%%%%%%%%%%%%%%%%%%%%%%%
	%On définit encore quelques trucs
	%%%%%%%%%%%%%%%%%%%%%%%%%%%%
%Pour commencer la bibliographie quoi c'est la fin mais là on commence par le nom du .bib de biblio
	\bib{library}
%le type du rapport : C’est le type de rapport (le grand titre). Par exemple : Rapport préliminaire,Rapport de stage,...
	\typerapport{Rapport de stage}
%l'année scolaire du travail
	\annee{20XX-20XX}	
%Entrer ici le nom des auteurs :
	\auteur{Your Name}
%Le Titre du document en question
	\titre{Title}
%Le Titre comme il doit apparaitre sur la page de résumé
	\titreresum{Short title}
%Le Titre du document en anglais
	\titrea{English title}
%Le dessin de titre image utilisé sur la page de garde fun
	\dessintitre{Title image}

%Période de stage et Date de soutenance peut être mis au jour de la compilation par la commande \today
	\per{Du 06 Avril au 30 juin 2010}
	\sout{Mercredi 16 Juin 2010}
%Nom du maitre de stage
	\maitrestage{Names master of stage}
%Nom du correspondant universitaire :
	\tuteurstage{corresp univ} 
%Vos mots clés du rapport :
	\motscles{french keywords}
%Les même en anglais :
	\keywords{english keywords}
%Pour entrer votre résumé taper votre texte dans le document 7-resume du dossier partie.
	\resum{	%Ici c'est pour le résumé :
\begin{singlespace}
Du système a-parental, où aucun soin post-natal n'est assuré au système communal, une multitude de systèmes d'appariements coexistent parmi les animaux. Chez les Oiseaux, l'investissement parental est majoritairement biparental. Ceci entraîne des conflits, chacun ayant intérêt à minimiser les coûts associés à la reproduction. Ce phénomène est d'autant plus vrai au cours de la post-garde, période de nourrissage du poussin, particulièrement coûteuse énergétiquement et donc à l'origine de choix reproducteurs forts. 

Cette étude porte sur l'analyse de données obtenues, chez le Manchot royal, lors de la saison de reproduction 2004-2005 et issues du système d'identification automatique \textsc{Antavia}, tout au long de la post-garde. Trois phases ont été distinguées 1- Crèche 1, de l'émancipation thermique à l'hiver 2- Crèche 2, phase hivernale 3- Crèche 3, phase de re-nourrissage jusqu'à l'envol du poussin. Les mâles réalisent un nombre de retours à terre supérieur aux femelles sur l'ensemble de la post-garde, et plus particulièrement au cours des deux premières crèches. Au cours de la troisième, le nombre de retours à terre ne dépend plus du sexe, mais de l'arrêt du nourrissage par un parent. Malgré une majorité d'arrêts simultanés, 28\% des poussins ont reçu des soins monoparentaux au cours de leur dernier mois d'élevage. Comme prévu par la théorie des jeux, une compensation partielle du parent restant suite à la désertion du partenaire a été mise en évidence. De plus, nous suggérons que les mâles jouent un rôle majeur dans le nourrissage du poussin, leur fréquence de nourrissages influençant l'issue de la reproduction.

\end{singlespace}}
%Le résumé en anglais est à taper dans le document 8-abstract
	\abstracts{%Tapé ici votre résumé directement simplement.
\begin{singlespace}
From aparental care to communal breeding, a diversity of partnership can be found amongst animals. Birds are unique among vertebrates in that biparental care is the norm. But biparental care goes along with potential conflicts between patners, where both try to minimize cost of reproduction but must cooperate at the same time. In many bird species, the post-guard phase is an energetically costly period when crucial decisions between costs and benefits have to be made.  
This study on King Penguins carries out the analysis of \textsc{Antavia}'s data, an automatic identification system, during post-guard phases of the reproductive cycle of 40 pairs in 2004-2005. Three different periods were distinguished 1\-- Crèche 1, from thermal emancipation to winter, 2\-- Crèche 2, winter period, 3\-- Crèche 3, refeeding phase until fledging.
Males completed more shifts ashore than females on average during the whole post-guard. This held true in the first two crèches. In the third one, the number of sojourns ashore is not sex-specific anymore but dependent from the cessation in food provisioning. Despite a majority of simultaneous stops, 28\% of chicks received mono-parental care during their last rearing month.  As predicted by the game theory, we observed a partial compensation of the remaining parent. Finally, we suggest that males play a key role in chick provisioning, as only males’ frequency affected the reproductive issue. 

\end{singlespace} }
%Logo de l'entreprise ou du Labo ou est réalisé le stage (pour la page de titre) placer entre les accolades le nom du fichier image.
	\logoentrep{logo}
%Nom de la structure d'accueil
	\nomentrep{name entreprise}
%Adresse de la structure d'accueil
	\addrentrep{adresse}
%Logo de l'université si vous utiliser les dossiers fourni pas besoin de changer quoique ce soit à ce niveau là 
	\logouniv{logo univ}
%Nom de l'université
	\nomuniv{name univ}
%Option d'étude = master dans lequel vous êtes :
	\optionetude{name course}