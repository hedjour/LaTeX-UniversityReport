%Ici c'est pour le résumé :
\begin{singlespace}
Du système a-parental, où aucun soin post-natal n'est assuré au système communal, une multitude de systèmes d'appariements coexistent parmi les animaux. Chez les Oiseaux, l'investissement parental est majoritairement biparental. Ceci entraîne des conflits, chacun ayant intérêt à minimiser les coûts associés à la reproduction. Ce phénomène est d'autant plus vrai au cours de la post-garde, période de nourrissage du poussin, particulièrement coûteuse énergétiquement et donc à l'origine de choix reproducteurs forts. 

Cette étude porte sur l'analyse de données obtenues, chez le Manchot royal, lors de la saison de reproduction 2004-2005 et issues du système d'identification automatique \textsc{Antavia}, tout au long de la post-garde. Trois phases ont été distinguées 1- Crèche 1, de l'émancipation thermique à l'hiver 2- Crèche 2, phase hivernale 3- Crèche 3, phase de re-nourrissage jusqu'à l'envol du poussin. Les mâles réalisent un nombre de retours à terre supérieur aux femelles sur l'ensemble de la post-garde, et plus particulièrement au cours des deux premières crèches. Au cours de la troisième, le nombre de retours à terre ne dépend plus du sexe, mais de l'arrêt du nourrissage par un parent. Malgré une majorité d'arrêts simultanés, 28\% des poussins ont reçu des soins monoparentaux au cours de leur dernier mois d'élevage. Comme prévu par la théorie des jeux, une compensation partielle du parent restant suite à la désertion du partenaire a été mise en évidence. De plus, nous suggérons que les mâles jouent un rôle majeur dans le nourrissage du poussin, leur fréquence de nourrissages influençant l'issue de la reproduction.

\end{singlespace}